
\documentclass{article}
\usepackage{amsmath, amssymb, hyperref}

\title{Quantum Superposition in Black Holes: A Proposed Resolution to the Information Paradox via Delayed Wavefunction Collapse}
\author{Eric Steven Tabor}
\date{February 15, 2025 (to be updated upon GitHub posting)}

\begin{document}

\maketitle

\begin{abstract}
This paper proposes a novel resolution to the black hole information paradox by integrating quantum superposition, entanglement, and event horizons into a unified framework. Traditional theories suggest that information falling into a black hole is either lost (violating quantum mechanics) or encoded in Hawking Radiation (leading to paradoxes in quantum gravity).

We hypothesize that all particles inside a black hole remain in quantum superposition due to the lack of external measurement. If true, this suggests that information is not lost but delayed in its wavefunction collapse until the black hole fully evaporates. At the final stage of Hawking Radiation, a global collapse of all entangled states occurs, restoring quantum information back into the universe.

Additionally, we explore the possibility of artificial event horizons—synthetic structures that could mimic black hole physics to stabilize entanglement for quantum information storage, teleportation, or even space-time manipulation. This paper outlines theoretical foundations, mathematical implications, and potential experimental approaches to validate or falsify this hypothesis.
\end{abstract}

\section{Introduction: The Information Paradox and the Limits of Current Theories}

Black holes pose one of the greatest challenges to modern physics, acting as a bridge between quantum mechanics and general relativity. One of the most significant unsolved problems is the black hole information paradox—what happens to information that falls past the event horizon?

\subsection{The Standard View}
\begin{itemize}
    \item General relativity suggests that information falling into a black hole is lost forever inside the singularity.
    \item Quantum mechanics, however, demands that information must be preserved (unitarity).
    \item Hawking Radiation (1974) suggests black holes evaporate over time, but it appears to radiate random thermal energy, not carrying information.
    \item If information is lost, it violates quantum mechanics—if it is preserved, we must explain how and where it remains encoded.
\end{itemize}

\subsection{The Hypothesis: Superposition and Delayed Wavefunction Collapse}
This paper presents a new possibility:
\begin{enumerate}
    \item All particles inside a black hole exist in quantum superposition due to the absence of measurement.
    \item Hawking Radiation slowly removes entangled information, but the wavefunction does not collapse until complete evaporation.
    \item At the final evaporation stage, all entangled states collapse at once, restoring the lost information.
\end{enumerate}

This perspective suggests that black holes are not information-destroying objects, but quantum delay systems that store quantum states in entanglement until they are “released” upon complete evaporation.

\section{Mathematical Foundation of Black Hole Superposition and Entanglement}

\subsection{Persistence of Superposition Inside a Black Hole}
Quantum mechanics states that a system remains in superposition until measured. The time evolution of a quantum system is given by Schrödinger’s equation:

\begin{equation}
i\hbar \frac{d}{dt} \Psi = \hat{H} \Psi
\end{equation}

where:
\begin{itemize}
    \item \( \Psi \) is the quantum state (wavefunction),
    \item \( \hbar \) is the reduced Planck’s constant,
    \item \( \hat{H} \) is the Hamiltonian (total energy operator).
\end{itemize}

In a black hole, there is no external observer to measure \( \Psi \), so the wavefunction remains uncollapsed. Thus, quantum states inside the black hole remain entangled indefinitely.

\subsection{Entanglement and Hawking Radiation}

Hawking Radiation follows the formula:

\begin{equation}
T_H = \frac{\hbar c^3}{8\pi G M k_B}
\end{equation}

where:
\begin{itemize}
    \item \( T_H \) is the Hawking temperature,
    \item \( M \) is the black hole’s mass,
    \item \( G \) is the gravitational constant,
    \item \( k_B \) is the Boltzmann constant,
    \item \( c \) is the speed of light.
\end{itemize}

The entanglement entropy follows:

\begin{equation}
S_{\text{ent}} = \frac{k_B A}{4G\hbar}
\end{equation}

where \( A \) is the event horizon surface area. This implies that Hawking Radiation carries away quantum information but does not collapse the wavefunction inside the black hole.

\subsection{The Final Collapse at the Moment of Evaporation}

If superposition persists inside the black hole, then at the final moment of evaporation, the system undergoes a global wavefunction collapse:

\begin{equation}
\Psi_{\text{BH}} = \sum_i c_i \psi_i
\end{equation}

At full evaporation:

\begin{equation}
\Psi_{\text{BH}} \rightarrow \sum_i c_i |i\rangle_{\text{Hawking}}
\end{equation}

ensuring that information is preserved and released back into the universe.

\section{Artificial Event Horizons: Engineering a Controlled Quantum System}
\begin{itemize}
    \item Optical Analog Black Holes
    \item Metamaterials to Manipulate Waveforms
    \item Quantum Cavity Systems
\end{itemize}

\section{Experimental Approaches \& Potential Tests}
\begin{itemize}
    \item Gravitational Wave Data Analysis
    \item High-Energy Quantum Teleportation Experiments
    \item Hawking Radiation Information Encoding
\end{itemize}

\section{Conclusion \& Future Research Directions}
This hypothesis suggests that black holes preserve quantum information in an uncollapsed state until complete evaporation. Artificial event horizons could provide an experimental framework for quantum information storage, teleportation, and space-time manipulation.

\end{document}
